% Options for packages loaded elsewhere
% Options for packages loaded elsewhere
\PassOptionsToPackage{unicode}{hyperref}
\PassOptionsToPackage{hyphens}{url}
\PassOptionsToPackage{dvipsnames,svgnames,x11names}{xcolor}
%
\documentclass[
  letterpaper,
  DIV=11,
  numbers=noendperiod]{scrartcl}
\usepackage{xcolor}
\usepackage{amsmath,amssymb}
\setcounter{secnumdepth}{-\maxdimen} % remove section numbering
\usepackage{iftex}
\ifPDFTeX
  \usepackage[T1]{fontenc}
  \usepackage[utf8]{inputenc}
  \usepackage{textcomp} % provide euro and other symbols
\else % if luatex or xetex
  \usepackage{unicode-math} % this also loads fontspec
  \defaultfontfeatures{Scale=MatchLowercase}
  \defaultfontfeatures[\rmfamily]{Ligatures=TeX,Scale=1}
\fi
\usepackage{lmodern}
\ifPDFTeX\else
  % xetex/luatex font selection
\fi
% Use upquote if available, for straight quotes in verbatim environments
\IfFileExists{upquote.sty}{\usepackage{upquote}}{}
\IfFileExists{microtype.sty}{% use microtype if available
  \usepackage[]{microtype}
  \UseMicrotypeSet[protrusion]{basicmath} % disable protrusion for tt fonts
}{}
\makeatletter
\@ifundefined{KOMAClassName}{% if non-KOMA class
  \IfFileExists{parskip.sty}{%
    \usepackage{parskip}
  }{% else
    \setlength{\parindent}{0pt}
    \setlength{\parskip}{6pt plus 2pt minus 1pt}}
}{% if KOMA class
  \KOMAoptions{parskip=half}}
\makeatother
% Make \paragraph and \subparagraph free-standing
\makeatletter
\ifx\paragraph\undefined\else
  \let\oldparagraph\paragraph
  \renewcommand{\paragraph}{
    \@ifstar
      \xxxParagraphStar
      \xxxParagraphNoStar
  }
  \newcommand{\xxxParagraphStar}[1]{\oldparagraph*{#1}\mbox{}}
  \newcommand{\xxxParagraphNoStar}[1]{\oldparagraph{#1}\mbox{}}
\fi
\ifx\subparagraph\undefined\else
  \let\oldsubparagraph\subparagraph
  \renewcommand{\subparagraph}{
    \@ifstar
      \xxxSubParagraphStar
      \xxxSubParagraphNoStar
  }
  \newcommand{\xxxSubParagraphStar}[1]{\oldsubparagraph*{#1}\mbox{}}
  \newcommand{\xxxSubParagraphNoStar}[1]{\oldsubparagraph{#1}\mbox{}}
\fi
\makeatother

\usepackage{color}
\usepackage{fancyvrb}
\newcommand{\VerbBar}{|}
\newcommand{\VERB}{\Verb[commandchars=\\\{\}]}
\DefineVerbatimEnvironment{Highlighting}{Verbatim}{commandchars=\\\{\}}
% Add ',fontsize=\small' for more characters per line
\usepackage{framed}
\definecolor{shadecolor}{RGB}{241,243,245}
\newenvironment{Shaded}{\begin{snugshade}}{\end{snugshade}}
\newcommand{\AlertTok}[1]{\textcolor[rgb]{0.68,0.00,0.00}{#1}}
\newcommand{\AnnotationTok}[1]{\textcolor[rgb]{0.37,0.37,0.37}{#1}}
\newcommand{\AttributeTok}[1]{\textcolor[rgb]{0.40,0.45,0.13}{#1}}
\newcommand{\BaseNTok}[1]{\textcolor[rgb]{0.68,0.00,0.00}{#1}}
\newcommand{\BuiltInTok}[1]{\textcolor[rgb]{0.00,0.23,0.31}{#1}}
\newcommand{\CharTok}[1]{\textcolor[rgb]{0.13,0.47,0.30}{#1}}
\newcommand{\CommentTok}[1]{\textcolor[rgb]{0.37,0.37,0.37}{#1}}
\newcommand{\CommentVarTok}[1]{\textcolor[rgb]{0.37,0.37,0.37}{\textit{#1}}}
\newcommand{\ConstantTok}[1]{\textcolor[rgb]{0.56,0.35,0.01}{#1}}
\newcommand{\ControlFlowTok}[1]{\textcolor[rgb]{0.00,0.23,0.31}{\textbf{#1}}}
\newcommand{\DataTypeTok}[1]{\textcolor[rgb]{0.68,0.00,0.00}{#1}}
\newcommand{\DecValTok}[1]{\textcolor[rgb]{0.68,0.00,0.00}{#1}}
\newcommand{\DocumentationTok}[1]{\textcolor[rgb]{0.37,0.37,0.37}{\textit{#1}}}
\newcommand{\ErrorTok}[1]{\textcolor[rgb]{0.68,0.00,0.00}{#1}}
\newcommand{\ExtensionTok}[1]{\textcolor[rgb]{0.00,0.23,0.31}{#1}}
\newcommand{\FloatTok}[1]{\textcolor[rgb]{0.68,0.00,0.00}{#1}}
\newcommand{\FunctionTok}[1]{\textcolor[rgb]{0.28,0.35,0.67}{#1}}
\newcommand{\ImportTok}[1]{\textcolor[rgb]{0.00,0.46,0.62}{#1}}
\newcommand{\InformationTok}[1]{\textcolor[rgb]{0.37,0.37,0.37}{#1}}
\newcommand{\KeywordTok}[1]{\textcolor[rgb]{0.00,0.23,0.31}{\textbf{#1}}}
\newcommand{\NormalTok}[1]{\textcolor[rgb]{0.00,0.23,0.31}{#1}}
\newcommand{\OperatorTok}[1]{\textcolor[rgb]{0.37,0.37,0.37}{#1}}
\newcommand{\OtherTok}[1]{\textcolor[rgb]{0.00,0.23,0.31}{#1}}
\newcommand{\PreprocessorTok}[1]{\textcolor[rgb]{0.68,0.00,0.00}{#1}}
\newcommand{\RegionMarkerTok}[1]{\textcolor[rgb]{0.00,0.23,0.31}{#1}}
\newcommand{\SpecialCharTok}[1]{\textcolor[rgb]{0.37,0.37,0.37}{#1}}
\newcommand{\SpecialStringTok}[1]{\textcolor[rgb]{0.13,0.47,0.30}{#1}}
\newcommand{\StringTok}[1]{\textcolor[rgb]{0.13,0.47,0.30}{#1}}
\newcommand{\VariableTok}[1]{\textcolor[rgb]{0.07,0.07,0.07}{#1}}
\newcommand{\VerbatimStringTok}[1]{\textcolor[rgb]{0.13,0.47,0.30}{#1}}
\newcommand{\WarningTok}[1]{\textcolor[rgb]{0.37,0.37,0.37}{\textit{#1}}}

\usepackage{longtable,booktabs,array}
\usepackage{calc} % for calculating minipage widths
% Correct order of tables after \paragraph or \subparagraph
\usepackage{etoolbox}
\makeatletter
\patchcmd\longtable{\par}{\if@noskipsec\mbox{}\fi\par}{}{}
\makeatother
% Allow footnotes in longtable head/foot
\IfFileExists{footnotehyper.sty}{\usepackage{footnotehyper}}{\usepackage{footnote}}
\makesavenoteenv{longtable}
\usepackage{graphicx}
\makeatletter
\newsavebox\pandoc@box
\newcommand*\pandocbounded[1]{% scales image to fit in text height/width
  \sbox\pandoc@box{#1}%
  \Gscale@div\@tempa{\textheight}{\dimexpr\ht\pandoc@box+\dp\pandoc@box\relax}%
  \Gscale@div\@tempb{\linewidth}{\wd\pandoc@box}%
  \ifdim\@tempb\p@<\@tempa\p@\let\@tempa\@tempb\fi% select the smaller of both
  \ifdim\@tempa\p@<\p@\scalebox{\@tempa}{\usebox\pandoc@box}%
  \else\usebox{\pandoc@box}%
  \fi%
}
% Set default figure placement to htbp
\def\fps@figure{htbp}
\makeatother





\setlength{\emergencystretch}{3em} % prevent overfull lines

\providecommand{\tightlist}{%
  \setlength{\itemsep}{0pt}\setlength{\parskip}{0pt}}



 


\KOMAoption{captions}{tableheading}
\makeatletter
\@ifpackageloaded{tcolorbox}{}{\usepackage[skins,breakable]{tcolorbox}}
\@ifpackageloaded{fontawesome5}{}{\usepackage{fontawesome5}}
\definecolor{quarto-callout-color}{HTML}{909090}
\definecolor{quarto-callout-note-color}{HTML}{0758E5}
\definecolor{quarto-callout-important-color}{HTML}{CC1914}
\definecolor{quarto-callout-warning-color}{HTML}{EB9113}
\definecolor{quarto-callout-tip-color}{HTML}{00A047}
\definecolor{quarto-callout-caution-color}{HTML}{FC5300}
\definecolor{quarto-callout-color-frame}{HTML}{acacac}
\definecolor{quarto-callout-note-color-frame}{HTML}{4582ec}
\definecolor{quarto-callout-important-color-frame}{HTML}{d9534f}
\definecolor{quarto-callout-warning-color-frame}{HTML}{f0ad4e}
\definecolor{quarto-callout-tip-color-frame}{HTML}{02b875}
\definecolor{quarto-callout-caution-color-frame}{HTML}{fd7e14}
\makeatother
\makeatletter
\@ifpackageloaded{caption}{}{\usepackage{caption}}
\AtBeginDocument{%
\ifdefined\contentsname
  \renewcommand*\contentsname{Table of contents}
\else
  \newcommand\contentsname{Table of contents}
\fi
\ifdefined\listfigurename
  \renewcommand*\listfigurename{List of Figures}
\else
  \newcommand\listfigurename{List of Figures}
\fi
\ifdefined\listtablename
  \renewcommand*\listtablename{List of Tables}
\else
  \newcommand\listtablename{List of Tables}
\fi
\ifdefined\figurename
  \renewcommand*\figurename{Figure}
\else
  \newcommand\figurename{Figure}
\fi
\ifdefined\tablename
  \renewcommand*\tablename{Table}
\else
  \newcommand\tablename{Table}
\fi
}
\@ifpackageloaded{float}{}{\usepackage{float}}
\floatstyle{ruled}
\@ifundefined{c@chapter}{\newfloat{codelisting}{h}{lop}}{\newfloat{codelisting}{h}{lop}[chapter]}
\floatname{codelisting}{Listing}
\newcommand*\listoflistings{\listof{codelisting}{List of Listings}}
\makeatother
\makeatletter
\makeatother
\makeatletter
\@ifpackageloaded{caption}{}{\usepackage{caption}}
\@ifpackageloaded{subcaption}{}{\usepackage{subcaption}}
\makeatother
\usepackage{bookmark}
\IfFileExists{xurl.sty}{\usepackage{xurl}}{} % add URL line breaks if available
\urlstyle{same}
\hypersetup{
  pdftitle={Chapter 5: STRUCTURED TYPES AND MUTABILITY},
  pdfauthor={CS 101 - Fall 2025},
  colorlinks=true,
  linkcolor={blue},
  filecolor={Maroon},
  citecolor={Blue},
  urlcolor={Blue},
  pdfcreator={LaTeX via pandoc}}


\title{Chapter 5: STRUCTURED TYPES AND MUTABILITY}
\usepackage{etoolbox}
\makeatletter
\providecommand{\subtitle}[1]{% add subtitle to \maketitle
  \apptocmd{\@title}{\par {\large #1 \par}}{}{}
}
\makeatother
\subtitle{Python challenges}
\author{CS 101 - Fall 2025}
\date{}
\begin{document}
\maketitle


\section{Putting It All Together: The Grand Finale!
🎉}\label{putting-it-all-together-the-grand-finale}

\begin{tcolorbox}[enhanced jigsaw, colbacktitle=quarto-callout-tip-color!10!white, toptitle=1mm, titlerule=0mm, breakable, toprule=.15mm, bottomrule=.15mm, opacityback=0, coltitle=black, leftrule=.75mm, colframe=quarto-callout-tip-color-frame, bottomtitle=1mm, opacitybacktitle=0.6, left=2mm, colback=white, arc=.35mm, title={Summary of Our Python Journey}, rightrule=.15mm]

\begin{itemize}
\tightlist
\item
  👯‍♀️ \textbf{Tuples}: Immutable, ordered data
\item
  🔢 \textbf{Ranges}: Efficient number sequences
\item
  🔄 \textbf{Lists}: Flexible, mutable collections
\item
  👥 \textbf{List Cloning}: Creating independent copies
\item
  ✨ \textbf{List Comprehensions}: Elegant one-line creation
\item
  📦 \textbf{Nested Lists}: Organizing hierarchical data
\item
  🎯 \textbf{2D Lists}: Grid-based data structures
\item
  🔧 \textbf{Higher-Order Operations}: Functional programming tools
\item
  👨‍👩‍👧‍👦 \textbf{Sequence Types}: Common operations across types
\item
  🌟 \textbf{Sets}: Unique collections with math operations
\end{itemize}

\end{tcolorbox}

🐍✨🚀💎

🎓📚💻

🎉🌟⚡

Now you're equipped with Python's most powerful data structures! 🐍✨

\begin{center}\rule{0.5\linewidth}{0.5pt}\end{center}

\section{Practice Time! 💪}\label{practice-time}

\begin{tcolorbox}[enhanced jigsaw, colbacktitle=quarto-callout-warning-color!10!white, toptitle=1mm, titlerule=0mm, breakable, toprule=.15mm, bottomrule=.15mm, opacityback=0, coltitle=black, leftrule=.75mm, colframe=quarto-callout-warning-color-frame, bottomtitle=1mm, opacitybacktitle=0.6, left=2mm, colback=white, arc=.35mm, title={Your Turn to Shine!}, rightrule=.15mm]

Let's practice with focused, manageable exercises! Each challenge takes
5-10 minutes and focuses on one key concept.

\textbf{Choose your challenge level:} - � \textbf{Basic}: Single concept
practice - 🟡 \textbf{Intermediate}: Combine 2-3 concepts - 🔴
\textbf{Advanced}: Combine multiple concepts

\textbf{Remember:} Start with basic challenges and work your way up! 🚀

\end{tcolorbox}

💪🎨🧠

👨‍💻👩‍💻🎯

🌈✨🎉

\begin{center}\rule{0.5\linewidth}{0.5pt}\end{center}

\subsection{Challenge 1: Tuple Practice
🟢}\label{challenge-1-tuple-practice}

\begin{tcolorbox}[enhanced jigsaw, colbacktitle=quarto-callout-tip-color!10!white, toptitle=1mm, titlerule=0mm, breakable, toprule=.15mm, bottomrule=.15mm, opacityback=0, coltitle=black, leftrule=.75mm, colframe=quarto-callout-tip-color-frame, bottomtitle=1mm, opacitybacktitle=0.6, left=2mm, colback=white, arc=.35mm, title={Basic Tuple Operations (5 minutes)}, rightrule=.15mm]

Practice creating and using tuples for storing related data.

\end{tcolorbox}

\begin{tcolorbox}[enhanced jigsaw, colbacktitle=quarto-callout-important-color!10!white, toptitle=1mm, titlerule=0mm, breakable, toprule=.15mm, bottomrule=.15mm, opacityback=0, coltitle=black, leftrule=.75mm, colframe=quarto-callout-important-color-frame, bottomtitle=1mm, opacitybacktitle=0.6, left=2mm, colback=white, arc=.35mm, title={Task}, rightrule=.15mm]

\begin{Shaded}
\begin{Highlighting}[]
\CommentTok{\# Given data}
\NormalTok{student1 }\OperatorTok{=}\NormalTok{ (}\StringTok{"Alice"}\NormalTok{, }\DecValTok{85}\NormalTok{, }\StringTok{"Computer Science"}\NormalTok{)}
\NormalTok{student2 }\OperatorTok{=}\NormalTok{ (}\StringTok{"Bob"}\NormalTok{, }\DecValTok{92}\NormalTok{, }\StringTok{"Mathematics"}\NormalTok{)}
\NormalTok{student3 }\OperatorTok{=}\NormalTok{ (}\StringTok{"Carol"}\NormalTok{, }\DecValTok{78}\NormalTok{, }\StringTok{"Physics"}\NormalTok{)}

\CommentTok{\# Your tasks (complete each line):}
\CommentTok{\# 1. Create a tuple with all three students}
\NormalTok{all\_students }\OperatorTok{=} \CommentTok{\# Your code here}

\CommentTok{\# 2. Get Alice\textquotesingle{}s grade (second element of first tuple)}
\NormalTok{alice\_grade }\OperatorTok{=} \CommentTok{\# Your code here}

\CommentTok{\# 3. Get all student names in a list}
\NormalTok{names }\OperatorTok{=} \CommentTok{\# Your code here}

\CommentTok{\# 4. Count how many students have grades above 80}
\NormalTok{high\_grades }\OperatorTok{=} \CommentTok{\# Your code here}
\end{Highlighting}
\end{Shaded}

\end{tcolorbox}

\begin{center}\rule{0.5\linewidth}{0.5pt}\end{center}

\subsection{Challenge 2: List Comprehension Magic
🟡}\label{challenge-2-list-comprehension-magic}

\begin{tcolorbox}[enhanced jigsaw, colbacktitle=quarto-callout-tip-color!10!white, toptitle=1mm, titlerule=0mm, breakable, toprule=.15mm, bottomrule=.15mm, opacityback=0, coltitle=black, leftrule=.75mm, colframe=quarto-callout-tip-color-frame, bottomtitle=1mm, opacitybacktitle=0.6, left=2mm, colback=white, arc=.35mm, title={List Comprehensions (7 minutes)}, rightrule=.15mm]

Create lists efficiently using comprehension syntax.

\end{tcolorbox}

\begin{tcolorbox}[enhanced jigsaw, colbacktitle=quarto-callout-important-color!10!white, toptitle=1mm, titlerule=0mm, breakable, toprule=.15mm, bottomrule=.15mm, opacityback=0, coltitle=black, leftrule=.75mm, colframe=quarto-callout-important-color-frame, bottomtitle=1mm, opacitybacktitle=0.6, left=2mm, colback=white, arc=.35mm, title={Task}, rightrule=.15mm]

\begin{Shaded}
\begin{Highlighting}[]
\CommentTok{\# Given data}
\NormalTok{numbers }\OperatorTok{=}\NormalTok{ [}\DecValTok{1}\NormalTok{, }\DecValTok{2}\NormalTok{, }\DecValTok{3}\NormalTok{, }\DecValTok{4}\NormalTok{, }\DecValTok{5}\NormalTok{, }\DecValTok{6}\NormalTok{, }\DecValTok{7}\NormalTok{, }\DecValTok{8}\NormalTok{, }\DecValTok{9}\NormalTok{, }\DecValTok{10}\NormalTok{]}
\NormalTok{words }\OperatorTok{=}\NormalTok{ [}\StringTok{"apple"}\NormalTok{, }\StringTok{"banana"}\NormalTok{, }\StringTok{"cherry"}\NormalTok{, }\StringTok{"date"}\NormalTok{, }\StringTok{"elderberry"}\NormalTok{]}

\CommentTok{\# Your tasks:}
\CommentTok{\# 1. Create a list of squares for even numbers only}
\NormalTok{even\_squares }\OperatorTok{=} \CommentTok{\# Your code here}

\CommentTok{\# 2. Create a list of words with more than 5 letters, in uppercase}
\NormalTok{long\_words\_upper }\OperatorTok{=} \CommentTok{\# Your code here}

\CommentTok{\# 3. Create a list of numbers from 1{-}20 that are divisible by 3}
\NormalTok{divisible\_by\_3 }\OperatorTok{=} \CommentTok{\# Your code here}

\CommentTok{\# 4. Create a list with first letter of each word}
\NormalTok{first\_letters }\OperatorTok{=} \CommentTok{\# Your code here}
\end{Highlighting}
\end{Shaded}

\end{tcolorbox}

\begin{center}\rule{0.5\linewidth}{0.5pt}\end{center}

\subsection{Challenge 3: Set Operations
🟢}\label{challenge-3-set-operations}

\begin{tcolorbox}[enhanced jigsaw, colbacktitle=quarto-callout-tip-color!10!white, toptitle=1mm, titlerule=0mm, breakable, toprule=.15mm, bottomrule=.15mm, opacityback=0, coltitle=black, leftrule=.75mm, colframe=quarto-callout-tip-color-frame, bottomtitle=1mm, opacitybacktitle=0.6, left=2mm, colback=white, arc=.35mm, title={Working with Sets (5 minutes)}, rightrule=.15mm]

Practice set operations for finding unique elements and relationships.

\end{tcolorbox}

\begin{tcolorbox}[enhanced jigsaw, colbacktitle=quarto-callout-important-color!10!white, toptitle=1mm, titlerule=0mm, breakable, toprule=.15mm, bottomrule=.15mm, opacityback=0, coltitle=black, leftrule=.75mm, colframe=quarto-callout-important-color-frame, bottomtitle=1mm, opacitybacktitle=0.6, left=2mm, colback=white, arc=.35mm, title={Task}, rightrule=.15mm]

\begin{Shaded}
\begin{Highlighting}[]
\CommentTok{\# Given data}
\NormalTok{class\_a }\OperatorTok{=}\NormalTok{ \{}\StringTok{"Alice"}\NormalTok{, }\StringTok{"Bob"}\NormalTok{, }\StringTok{"Charlie"}\NormalTok{, }\StringTok{"David"}\NormalTok{, }\StringTok{"Eve"}\NormalTok{\}}
\NormalTok{class\_b }\OperatorTok{=}\NormalTok{ \{}\StringTok{"Charlie"}\NormalTok{, }\StringTok{"David"}\NormalTok{, }\StringTok{"Frank"}\NormalTok{, }\StringTok{"Grace"}\NormalTok{, }\StringTok{"Alice"}\NormalTok{\}}
\NormalTok{grades }\OperatorTok{=}\NormalTok{ [}\DecValTok{85}\NormalTok{, }\DecValTok{92}\NormalTok{, }\DecValTok{78}\NormalTok{, }\DecValTok{85}\NormalTok{, }\DecValTok{91}\NormalTok{, }\DecValTok{78}\NormalTok{, }\DecValTok{88}\NormalTok{, }\DecValTok{85}\NormalTok{, }\DecValTok{92}\NormalTok{, }\DecValTok{78}\NormalTok{]}

\CommentTok{\# Your tasks:}
\CommentTok{\# 1. Find students in both classes}
\NormalTok{both\_classes }\OperatorTok{=} \CommentTok{\# Your code here}

\CommentTok{\# 2. Find students only in class A}
\NormalTok{only\_class\_a }\OperatorTok{=} \CommentTok{\# Your code here}

\CommentTok{\# 3. Find all unique grades}
\NormalTok{unique\_grades }\OperatorTok{=} \CommentTok{\# Your code here}

\CommentTok{\# 4. Find total number of unique students}
\NormalTok{total\_students }\OperatorTok{=} \CommentTok{\# Your code here}
\end{Highlighting}
\end{Shaded}

\end{tcolorbox}

\begin{center}\rule{0.5\linewidth}{0.5pt}\end{center}

\subsection{Challenge 4: 2D List Basics
🟡}\label{challenge-4-2d-list-basics}

\begin{tcolorbox}[enhanced jigsaw, colbacktitle=quarto-callout-tip-color!10!white, toptitle=1mm, titlerule=0mm, breakable, toprule=.15mm, bottomrule=.15mm, opacityback=0, coltitle=black, leftrule=.75mm, colframe=quarto-callout-tip-color-frame, bottomtitle=1mm, opacitybacktitle=0.6, left=2mm, colback=white, arc=.35mm, title={Working with 2D Lists (8 minutes)}, rightrule=.15mm]

Practice creating and manipulating 2D lists for grid-based data.

\end{tcolorbox}

\textbf{Task:} Complete the following code

\begin{Shaded}
\begin{Highlighting}[]
\CommentTok{\# Create a 3x3 tic{-}tac{-}toe board}
\NormalTok{board }\OperatorTok{=}\NormalTok{ [}
\NormalTok{    [}\StringTok{\textquotesingle{} \textquotesingle{}}\NormalTok{, }\StringTok{\textquotesingle{} \textquotesingle{}}\NormalTok{, }\StringTok{\textquotesingle{} \textquotesingle{}}\NormalTok{],}
\NormalTok{    [}\StringTok{\textquotesingle{} \textquotesingle{}}\NormalTok{, }\StringTok{\textquotesingle{} \textquotesingle{}}\NormalTok{, }\StringTok{\textquotesingle{} \textquotesingle{}}\NormalTok{],}
\NormalTok{    [}\StringTok{\textquotesingle{} \textquotesingle{}}\NormalTok{, }\StringTok{\textquotesingle{} \textquotesingle{}}\NormalTok{, }\StringTok{\textquotesingle{} \textquotesingle{}}\NormalTok{]}
\NormalTok{]}

\CommentTok{\# Your tasks:}
\CommentTok{\# 1. Place \textquotesingle{}X\textquotesingle{} in the center (row 1, column 1)}
\CommentTok{\# Your code here}

\CommentTok{\# 2. Place \textquotesingle{}O\textquotesingle{} in top{-}left corner (row 0, column 0)}
\CommentTok{\# Your code here}

\CommentTok{\# 3. Create a list of all positions that are empty (\textquotesingle{} \textquotesingle{})}
\NormalTok{empty\_positions }\OperatorTok{=} \CommentTok{\# Your code here}

\CommentTok{\# 4. Check if the center row has any \textquotesingle{}X\textquotesingle{} in it}
\NormalTok{center\_has\_x }\OperatorTok{=} \CommentTok{\# Your code here}
\end{Highlighting}
\end{Shaded}

\begin{center}\rule{0.5\linewidth}{0.5pt}\end{center}

\subsection{Challenge 5: List Cloning Practice
🟡}\label{challenge-5-list-cloning-practice}

\begin{tcolorbox}[enhanced jigsaw, colbacktitle=quarto-callout-tip-color!10!white, toptitle=1mm, titlerule=0mm, breakable, toprule=.15mm, bottomrule=.15mm, opacityback=0, coltitle=black, leftrule=.75mm, colframe=quarto-callout-tip-color-frame, bottomtitle=1mm, opacitybacktitle=0.6, left=2mm, colback=white, arc=.35mm, title={Safe List Operations (6 minutes)}, rightrule=.15mm]

Practice cloning lists to avoid unwanted side effects.

\end{tcolorbox}

\begin{tcolorbox}[enhanced jigsaw, colbacktitle=quarto-callout-important-color!10!white, toptitle=1mm, titlerule=0mm, breakable, toprule=.15mm, bottomrule=.15mm, opacityback=0, coltitle=black, leftrule=.75mm, colframe=quarto-callout-important-color-frame, bottomtitle=1mm, opacitybacktitle=0.6, left=2mm, colback=white, arc=.35mm, title={Task}, rightrule=.15mm]

\begin{Shaded}
\begin{Highlighting}[]
\CommentTok{\# Original shopping list}
\NormalTok{original\_list }\OperatorTok{=}\NormalTok{ [}\StringTok{"apples"}\NormalTok{, }\StringTok{"bananas"}\NormalTok{, }\StringTok{"oranges"}\NormalTok{]}

\CommentTok{\# Your tasks:}
\CommentTok{\# 1. Create a proper copy of the original list}
\NormalTok{shopping\_copy }\OperatorTok{=} \CommentTok{\# Your code here}

\CommentTok{\# 2. Add "grapes" to the copy (original should remain unchanged)}
\CommentTok{\# Your code here}

\CommentTok{\# 3. Create another copy and remove "bananas" from it}
\NormalTok{fruit\_copy }\OperatorTok{=} \CommentTok{\# Your code here}
\CommentTok{\# Your code here}

\CommentTok{\# 4. Verify original list is unchanged}
\BuiltInTok{print}\NormalTok{(}\SpecialStringTok{f"Original: }\SpecialCharTok{\{}\NormalTok{original\_list}\SpecialCharTok{\}}\SpecialStringTok{"}\NormalTok{)  }\CommentTok{\# Should be ["apples", "bananas", "oranges"]}
\end{Highlighting}
\end{Shaded}

\end{tcolorbox}

\begin{center}\rule{0.5\linewidth}{0.5pt}\end{center}

\subsection{Challenge 6: Range and Map Practice
🟡}\label{challenge-6-range-and-map-practice}

\begin{tcolorbox}[enhanced jigsaw, colbacktitle=quarto-callout-tip-color!10!white, toptitle=1mm, titlerule=0mm, breakable, toprule=.15mm, bottomrule=.15mm, opacityback=0, coltitle=black, leftrule=.75mm, colframe=quarto-callout-tip-color-frame, bottomtitle=1mm, opacitybacktitle=0.6, left=2mm, colback=white, arc=.35mm, title={Higher-Order Functions (8 minutes)}, rightrule=.15mm]

Practice using ranges with map() and filter() functions.

\end{tcolorbox}

\begin{tcolorbox}[enhanced jigsaw, colbacktitle=quarto-callout-important-color!10!white, toptitle=1mm, titlerule=0mm, breakable, toprule=.15mm, bottomrule=.15mm, opacityback=0, coltitle=black, leftrule=.75mm, colframe=quarto-callout-important-color-frame, bottomtitle=1mm, opacitybacktitle=0.6, left=2mm, colback=white, arc=.35mm, title={Task}, rightrule=.15mm]

\begin{Shaded}
\begin{Highlighting}[]
\CommentTok{\# Given data}
\NormalTok{prices }\OperatorTok{=}\NormalTok{ [}\FloatTok{19.99}\NormalTok{, }\FloatTok{25.50}\NormalTok{, }\FloatTok{12.75}\NormalTok{, }\FloatTok{8.99}\NormalTok{, }\FloatTok{45.00}\NormalTok{, }\FloatTok{15.25}\NormalTok{]}

\CommentTok{\# Your tasks:}
\CommentTok{\# 1. Use range to create a list of numbers 0{-}9}
\NormalTok{numbers }\OperatorTok{=} \CommentTok{\# Your code here}

\CommentTok{\# 2. Use map() to add tax (8\%) to all prices}
\NormalTok{prices\_with\_tax }\OperatorTok{=} \CommentTok{\# Your code here}

\CommentTok{\# 3. Use filter() to find prices under $20}
\NormalTok{affordable\_prices }\OperatorTok{=} \CommentTok{\# Your code here}

\CommentTok{\# 4. Use map() to convert all prices to integers (rounded down)}
\NormalTok{rounded\_prices }\OperatorTok{=} \CommentTok{\# Your code here}
\end{Highlighting}
\end{Shaded}

\end{tcolorbox}

\begin{center}\rule{0.5\linewidth}{0.5pt}\end{center}

\subsection{Challenge 7: Combining Concepts
🔴}\label{challenge-7-combining-concepts}

\begin{tcolorbox}[enhanced jigsaw, colbacktitle=quarto-callout-tip-color!10!white, toptitle=1mm, titlerule=0mm, breakable, toprule=.15mm, bottomrule=.15mm, opacityback=0, coltitle=black, leftrule=.75mm, colframe=quarto-callout-tip-color-frame, bottomtitle=1mm, opacitybacktitle=0.6, left=2mm, colback=white, arc=.35mm, title={Integration Challenge (10 minutes)}, rightrule=.15mm]

Combine multiple concepts in a mini student database.

\end{tcolorbox}

\textbf{Task:} Complete the following code

\begin{Shaded}
\begin{Highlighting}[]
\CommentTok{\# Student data: (name, age, grades\_list)}
\NormalTok{students }\OperatorTok{=}\NormalTok{ [}
\NormalTok{    (}\StringTok{"Alice"}\NormalTok{, }\DecValTok{20}\NormalTok{, [}\DecValTok{85}\NormalTok{, }\DecValTok{92}\NormalTok{, }\DecValTok{78}\NormalTok{]),}
\NormalTok{    (}\StringTok{"Bob"}\NormalTok{, }\DecValTok{19}\NormalTok{, [}\DecValTok{90}\NormalTok{, }\DecValTok{88}\NormalTok{, }\DecValTok{95}\NormalTok{]),}
\NormalTok{    (}\StringTok{"Carol"}\NormalTok{, }\DecValTok{21}\NormalTok{, [}\DecValTok{76}\NormalTok{, }\DecValTok{82}\NormalTok{, }\DecValTok{85}\NormalTok{])}
\NormalTok{]}

\CommentTok{\# Your tasks:}
\CommentTok{\# 1. Create a set of all unique ages}
\NormalTok{unique\_ages }\OperatorTok{=} \CommentTok{\# Your code here}

\CommentTok{\# 2. Use list comprehension to get all student names}
\NormalTok{names }\OperatorTok{=} \CommentTok{\# Your code here}

\CommentTok{\# 3. Create a list of average grades for each student}
\NormalTok{averages }\OperatorTok{=} \CommentTok{\# Your code here}

\CommentTok{\# 4. Find students with average grade above 85}
\NormalTok{high\_performers }\OperatorTok{=} \CommentTok{\# Your code here}

\CommentTok{\# 5. Clone the students list and add a new student}
\NormalTok{students\_copy }\OperatorTok{=} \CommentTok{\# Your code here}
\CommentTok{\# Add ("David", 22, [88, 91, 87]) to the copy}
\end{Highlighting}
\end{Shaded}

\begin{center}\rule{0.5\linewidth}{0.5pt}\end{center}

\subsection{Quick Check Solutions 📝}\label{quick-check-solutions}

\begin{tcolorbox}[enhanced jigsaw, colbacktitle=quarto-callout-note-color!10!white, toptitle=1mm, titlerule=0mm, breakable, toprule=.15mm, bottomrule=.15mm, opacityback=0, coltitle=black, leftrule=.75mm, colframe=quarto-callout-note-color-frame, bottomtitle=1mm, opacitybacktitle=0.6, left=2mm, colback=white, arc=.35mm, title={Test Your Understanding}, rightrule=.15mm]

Before moving on, make sure you can explain:

\begin{itemize}
\tightlist
\item
  \textbf{Why} we use tuples vs lists
\item
  \textbf{How} list comprehensions make code cleaner
\item
  \textbf{When} to clone lists vs reference them
\item
  \textbf{What} makes sets useful for data analysis
\item
  \textbf{How} 2D lists represent grid data
\end{itemize}

\textbf{Hint:} If you can teach it to someone else, you've mastered it!
🎓

\end{tcolorbox}

Great job practicing! These building blocks will serve you well! 🐍✨




\end{document}
